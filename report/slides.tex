\documentclass{beamer}

\usepackage[italian]{babel}
\usepackage{array}
\usepackage{hyperref}

\usetheme{CambridgeUS}
\usecolortheme{orchid}
\setbeamertemplate{navigation symbols}{}

\title{Progetto di Ingegneria del software}
\author{Gruppo 13}
\institute{Università di Bologna}
\date{29 dicembre 2022}

\begin{document}

\begin{frame}
	\titlepage
\end{frame}

\begin{frame}{Indice}
	\tableofcontents
\end{frame}

\section{Introduzione}
\begin{frame}{Introduzione}
	Due categorie di studenti di Ingegneria del software all'Università di Miami
	svolgono lo stesso progetto in piccoli gruppi:
	\begin{itemize}
		\item i gruppi ``di controllo`` (C) fanno uso di metodi di sviluppo
		      tradizionali;
		\item i gruppi ``di trattamento`` (T) hanno studiato i \textbf{metodi
			      formali per lo sviluppo di programmi} .
	\end{itemize}
\end{frame}

\section{Approccio sperimentale}
\begin{frame}{Approccio sperimentale}
	\begin{table}
		\begin{tabular}{|p{0.5\linewidth}|c|c|}
			\hline
			                                            & C  & T  \\
			\hline
			Introduzione alla derivavione dei programmi & No & Sì \\
			\hline
			Semantica delle strutture di dati           & Sì & Sì \\
			\hline
			Progettazione orientata agli oggetti        & Sì & Sì \\
			\hline
			Analisi formale di programmi concorrenti    & No & Sì \\
			\hline
			Ingegneria del software                     & Sì & Sì \\
			\hline
			Laboratorio di sviluppo software            & Sì & Sì \\
			\hline
		\end{tabular}
		\caption{al di là della formazione specifica sui metodi formali, le due
			categorie condividono lo stesso curriculum}
	\end{table}

\end{frame}

\section{Il progetto}
\begin{frame}{Il progetto}
	Ogni gruppo deve realizzare un \textbf{sistema incorporato} per la gestione
	delle richieste effettuate nei confronti di \textbf{un ascensore}. È richiesto
	l'uso:
	\begin{itemize}
		\item dei principi della progettazione orientata agli oggetti
		      ;
		\item di \texttt{C++} per l'implementazione;
		\item di MFC per l'interfaccia grafica.
	\end{itemize}
\end{frame}

\section{Progettazione}
\begin{frame}
	\vfill
	\centering
	\usebeamerfont{title}
	Progettazione
	\vfill
\end{frame}

\subsection{Gruppi di controllo}
\begin{frame}{Progettazione: gruppi di controllo}
	Nessun gruppo ha consegnato diagrammi UML.
	\\~\\
	Tredici gruppi hanno consegnato prototipi. Di questi, nove hanno esibito
	codice sorgente.
	\\~\\
	Tutti i progetti sorgente consegnati mostrano un \textbf{grado di
		accoppiamento estremamente alto}: il sistema dell'ascensore e la gestione
	delle schermate convivono negli stessi moduli.
\end{frame}

\subsection{Gruppi di trattamento}
\begin{frame}{Progettazione: gruppi di trattamento}
	Tre gruppi su sei hanno consegnato diagrammi UML; di questi, uno solo presenta
	un alto grado di accoppiamento.
	\\~\\
	Quattro gruppi su sei presentano specifiche formali complete. Ad esempio:
	\begin{exampleblock}{Postcondizione per ogni operazione dell'ascensore}
		\[
			\begin{array}{l}
				(\exists p : Person \mid OnFloor(e, p)                                               \\
				\wedge \quad e.direction == p.direction : AddPerson(p))                              \\
				\wedge \quad (GoingDown(e) \rightarrow e.current\_{}floow := e.current\_{}floor - 1) \\
				\vee \quad GoingUp(e) \rightarrow e.current\_{}floor := e.current\_{}floor + 1)      \\
				\wedge \quad (\exists p : Person \mid p.ending\_{}floor ==                           \\
				e.current\_{}floor : RemovePerson(p)).
			\end{array}
		\]
	\end{exampleblock}
\end{frame}

\section{Implementazione}
\begin{frame}
	\vfill
	\centering
	\usebeamerfont{title}
	Implementazione
	\vfill
\end{frame}


\subsection{Gruppi di controllo}

\begin{frame}{Implementazione: gruppi di controllo}
	\begin{table}
		\begin{tabular}{|l|c|c|}
			\hline
			                            & Media & Dev. Std. \\
			\hline
			Linee totali                & 136   & 88        \\
			\hline
			Linee per la richiesta      & 67    & 83        \\
			\hline
			Linee per la pianificazione & 69    & 39        \\
			\hline
			Cicli                       & 7     & 6         \\
			\hline
			Condizioni                  & 16    & 10        \\
			\hline
			Casi                        & 30    & 20        \\
			\hline
			Livelli di innesto          & 4     & 1         \\
			\hline
		\end{tabular}
		\caption{analisi delle implementazioni dei gruppi di controllo. Nessuna
			implementazione passa tutti i test; tutte esibiscono un cattivo stile di
			programmazione.}
	\end{table}
\end{frame}

\subsection{Gruppi di trattamento}
\begin{frame}{Implementazione: gruppi di trattamento}
	\begin{table}
		\begin{tabular}{|l|c|c|}
			\hline
			                            & Media & Dev. Std. \\
			\hline
			Linee totali                & 117   & 44        \\
			\hline
			Linee per la richiesta      & 45    & 38        \\
			\hline
			Linee per la pianificazione & 72    & 38        \\
			\hline
			Cicli                       & 4     & 5         \\
			\hline
			Condizioni                  & 16    & 10        \\
			\hline
			Casi                        & 24    & 11        \\
			\hline
			Livelli di innesto          & 5     & 3         \\
			\hline
		\end{tabular}
		\caption{analisi delle implementazioni dei gruppi di trattamento. Ogni
			implementazione passa tutti i test; la qualità dello stile è variabile.}
	\end{table}
\end{frame}

\section{Una soluzione interamente formale}

\begin{frame}{Una soluzione interamente formale}
	Dopo la consegna, uno dei gruppi di trattamento ha ripetuto il progetto
	tramite un processo formale ancora più rigoroso:
	\begin{enumerate}
		\item un'analisi formale dei requisiti;
		\item un'implementazione intermedia in \texttt{Guarded Command Language}
		      ;
		\item una prova che l'implementazione intermedia rispettasse i requisiti
		      formali;
		\item una traduzione da \texttt{Guarded Command Language} a \texttt{C++}.
	\end{enumerate}
\end{frame}

\subsection{Progettazione}
\begin{frame}{Soluzione formale: progettazione}
	Gli errori incontrati in fase di progettazione erano spesso dovuti a
	specifiche incomplete, che sono state revisionate in corso d'opera.
	\\~\\
	Sono stati prodotti sia un diagramma UML che l'implementazione intermedia in
	\texttt{Guarded Command Language}.
\end{frame}

\subsection{Implementazione}
\begin{frame}{Soluzione formale: implementazione}
	Data l'implementazione intermedia, la fase di implementazione vera e propria
	è stata meccanica e banale.
	\\~\\
	Alla prima stesura del codice, gli unici errori riscontrati erano dovuti al
	codice non derivato dalle specifiche formali.
	\\~\\
	Dalla prima stesura al momento in cui il progetto ha passato ogni caso di
	prova, \textbf{il codice derivato dalle specifiche formali non ha mai subito
		modifiche}.
\end{frame}

\section{Osservazioni}

\begin{frame}{Osservazioni}
	In questo esperimento comparativo, i gruppi che hanno fatto uso di metodi
	formali ne hanno beneficiato enormemente in \textbf{correttezza} e
	considerevolmente in \textbf{qualità dello stile} di programmazione.
	\\~\\
	Solo l'ultimo gruppo, che ha derivato e verificato il proprio codice, ne ha
	anche guadagnato in concisione. Preso singolarmente, però, questo dato non è
	statisticamente significativo.
\end{frame}

\section{Conclusioni}
\begin{frame}{Conclusioni}
	Sobel e Clarkson, convalidando empiricamente l'ipotesi che l'analisi e la
	verifica formali siano determinanti per la capacità di risoluzione dei
	problemi, si augurano che queste pratiche siano incluse anche nei curricula di
	altri atenei.
\end{frame}

\end{document}
